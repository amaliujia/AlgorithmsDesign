The DRtree package keeps a DR index in an index file. Each index file is
composed by:
\begin{itemize}
	\item
	an index header page, which includes the following fields:
	\begin{enumerate}
		\item
		{\bf int dim}: the dimension of the indexed space;
		\item
		{\bf coord\_t ctype}: the coordinate type of indexed objects;
		\item
		{\bf int rectsize}: the size of a rectangle;
		\item
		{\bf int entrysize}: the size of an entry;
		\item
		{\bf int nodecard}: the number of entry slots in a node;
		\item
		{\bf int bmpsize}: the size of the bitmap used to indicate the
		usage of entry slots in a node;
		\item
		{\bf int rootpage}: the page number of the root node;
		\item
		{\bf int rootlevel}: the level of the root node in the index
		tree with leaf nodes at level 0;
	\end{enumerate}

	\item
	and a data page for each index node.\\
\end{itemize}

A DR index node is composed by:
\begin{itemize}
	\item
	a node header, which includes:
	\begin{enumerate}
		\item
		{\bf int level}: the level of a node in the index tree
		with leaf nodes at level 0;
		\item
		{\bf int count}: the number of used entry slots in a node;
	\end{enumerate}

	\item
	and a node buffer, which includes:
	\begin{enumerate}
		\item
		an array of entry slots with the array size to be {\em nodecard};
		\item
		an array of bytes(a bitmap) with each bit to indicate
		the usage of an entry slot, there may be some unused bits
		in the last byte.\\
	\end{enumerate}
\end{itemize}

An entry is composed by:
\begin{enumerate}
	\item
	the ID number of an indexed object if the entry is a leaf entry,
	the page number of the sub-tree root node otherwise;

	\item
	the covering rectangle of an indexed object or the sub-tree root node.\\
\end{enumerate}

A rectangle is composed by:
\begin{enumerate}
	\item
	an array of coordinates of all the lower sides of the rectangle
	in each dimension;
	\item
	an array of coordinates of all the higher sides of the rectangle
	in each dimension.
\end{enumerate}

